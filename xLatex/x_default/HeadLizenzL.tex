\documentclass[9pt]{scrartcl}
%%Sprache Deutsche zeichen
\usepackage[utf8]{inputenc}
\usepackage[utf8]{inputenc} 
\usepackage{lmodern} 

%%Deutsches Datum und weitere Angaben
\usepackage[ngerman]{babel}
%%Papersize
\usepackage[a4paper, top=30mm, bottom=30mm, left=0.8in, right=20mm,marginparwidth=68pt]{geometry}
%% Kleinere Überschriften
\usepackage[font=small,skip=0pt]{caption}
\usepackage{dirtree}
%%Packages Allgemein
\usepackage{paralist} %Itemize Enumerate in Paragraphs 
\usepackage{bold-extra} %% Bold smal Text
\usepackage{graphicx} %% More colors and settings for  inclúdegraphics
\usepackage{color} %% Color for Text 
\usepackage{eurosym} %% offizielles Euro symbol
\usepackage{framed} %% Rahmen erstellen \begin{framed}
\usepackage{epstopdf}
\usepackage{float} %% Improved floating images
\usepackage[pdftex]{hyperref} %% Hyperlinks
\usepackage{import} %% Better import (subimport)
\usepackage{array} %% Better Array and Tabular envirenment
\usepackage{enumerate} %% Aufzaehlung
\usepackage{tabularx} %% Better tables
\usepackage{datetime} %% bessere Datumsanzeige
\usepackage{microtype} %% Better reading
\usepackage{tikz} %% Vector graphics
\usepackage{multirow} %% Mehrspaltige Seiten
\usepackage{wrapfig}  %%% Texte um Bilder herrum
\usepackage{booktabs}  %% better Tables
\usepackage{tcolorbox}
\usepackage{multicol}
%%Packages Informatik
\usepackage{listings} %% Typesetting Syntax

\definecolor{pblue}{rgb}{0.13,0.13,1}
\definecolor{pgreen}{rgb}{0,0.5,0}
\definecolor{pred}{rgb}{0.9,0,0}
\definecolor{pgrey}{rgb}{0.46,0.45,0.48}

\usepackage{listings}
\lstset{language=Java,
	showspaces=false,
	showtabs=false,
	numbers=left,
	numberstyle=\tiny\color{black},
	stepnumber=2,
	tabsize=4,
	breaklines=true,
	showstringspaces=false,
	breakatwhitespace=true,
	commentstyle=\color{pgreen},
	keywordstyle=\color{pblue},
	stringstyle=\color{pred},
	basicstyle=\ttfamily,
	moredelim=[il][\textcolor{pgrey}]{$$},
	moredelim=[is][\textcolor{pgrey}]{\%\%}{\%\%}
}


\usepackage[os=win]{menukeys} %% einfügen von Tasten
\usepackage{struktex} %%struktogramme

%%Packages Mathematik
\usepackage{amsmath}
\usepackage{mathtools} %% optimize latex math
\usepackage[school]{pgf-umlcd}
%%Schule Allgemein

\usepackage{pgfplots} %% Plottel
\usepackage[
type={CC},
modifier={by-nc-sa},
version={4.0},
]{doclicense}
%%TKIS Libarys
\usetikzlibrary{trees}
%%% Test if needed
%\usepackage{url}
%\usepackage{ifthen}

\setlength\parindent{0pt} %% kein Einzug bei Zeilenumbruch

%%%%Header
\usepackage{scrlayer-scrpage}
\ihead{\tiny Informatik}
\chead{\tiny \thema}   %%%%% Set Header always
\ohead{\tiny \thepage}
\ofoot{\doclicenseImage[imagewidth=1.5cm]}
\ifoot{}
\cfoot{}
\newdateformat{mydate}{\_\_.\THEMONTH.\THEYEAR}

\newpairofpagestyles[scrheadings]{specialchapter}{
\ihead{}
\chead{}   %%%%% Set Header always
\ohead{}
\ofoot{\doclicenseImage[imagewidth=1.5cm]}
\ifoot{}
\cfoot{}
}

\newcommand{\marky}[1]{{\colorbox{black}{\color{white} #1}}}

\newcommand{\spielplanEintrag}[3]{\fbox{\parbox{12ex}{\centering {#1} gegen {#2}\newline {#3}}	}}
\newcommand{\schulkuerzel}{Freiherr-vom-Stein-Gymnasium Hamm\xspace}

\NewDocumentCommand{\anchormark}{O{0.15 cm} m O{0.05}}{%
	\tikz[overlay,remember picture,baseline=-0.5ex,xshift=#1] \node[draw,fill=black,circle,scale=#3] (#2) {};
}

